%  Template for ICASSP-2021 paper; to be used with:
%          spconf.sty  - ICASSP/ICIP LaTeX style file, and
%          IEEEbib.bst - IEEE bibliography style file.
% --------------------------------------------------------------------------
\documentclass{article}
\usepackage{spconf,amsmath,graphicx}

% Example definitions.
% --------------------
\def\x{{\mathbf x}}
\def\L{{\cal L}}

% Title.
% ------
\title{AIML 425 Assignemnt 1}
%
% Single address.
% ---------------
\name{Quan Zhao (Student ID: 300471028)}
%\name{Author(s) Name(s)\thanks{Thanks to XYZ agency for funding.}}
\address{Victoria University of Wellington}
%
% For example:
% ------------
%\address{School\\
%	Department\\
%	Address}
%
% Two addresses (uncomment and modify for two-address case).
% ----------------------------------------------------------
%\twoauthors
%  {A. Author-one, B. Author-two\sthanks{Thanks to XYZ agency for funding.}}
%	{School A-B\\
%	Department A-B\\
%	Address A-B}
%  {C. Author-three, D. Author-four\sthanks{The fourth author performed the work
%	while at ...}}
%	{School C-D\\
%	Department C-D\\
%	Address C-D}
%
\begin{document}
%\ninept
%
\maketitle
%
\section{Introduction}
\label{sec:intro}

Generative models are a subset of machine learning models designed to capture and mimic the underlying distribution of the data on which they are trained. They can produce new samples that resemble the input data but might not have been part of the original training set.

In this work, I will present my understanding of training and tuning generative models.

Three models will be trained:

1. The first model, $f1$, has an input that follows a 2D Gaussian distribution and produces an output following a 2D uniform distribution.

2. The second model, $f2$, takes the output of model $f1$ as its input and produces an output that follows a 2D Gaussian distribution.

3. The third model, $f3$, accepts an input that follows a 1D uniform distribution and generates an output following a 2D Gaussian distribution.

We have introduced experiments to discuss:

  - The effects of varying levels of L2 regularization on the weights for $f1$.
  
  - The implications of concatenating the two networks, $f1$ and $f2$.

Finally, we will showcase the point mapping from the input to the output of model $f3$.

% Below is an example of how to insert images. Delete the ``\vspace'' line,
% uncomment the preceding line ``\centerline...'' and replace ``imageX.ps''
% with a suitable PostScript file name.
% -------------------------------------------------------------------------
\begin{figure}[htb]

  \begin{minipage}[b]{1.0\linewidth}
    \centering
    \centerline{\includegraphics[width=8.5cm]{image1}}
  %  \vspace{2.0cm}
    \centerline{(a) Result 1}\medskip
  \end{minipage}
  %
  \begin{minipage}[b]{.48\linewidth}
    \centering
    \centerline{\includegraphics[width=4.0cm]{image3}}
  %  \vspace{1.5cm}
    \centerline{(b) Results 3}\medskip
  \end{minipage}
  \hfill
  \begin{minipage}[b]{0.48\linewidth}
    \centering
    \centerline{\includegraphics[width=4.0cm]{image4}}
  %  \vspace{1.5cm}
    \centerline{(c) Result 4}\medskip
  \end{minipage}
  %
  \caption{Example of placing a figure with experimental results.}
  \label{fig:res}
  %
  \end{figure}
  

\section{THEORY}
\label{sec:theory}

\subsection{MMD}
\label{ssec:mmd}

Maximum Mean Discrepancy (MMD) is a statistical test for determining whether two samples come from the same distribution. It's often used in the context of kernel methods and has applications in training generative models.

$ \text{MMD}^2 = E[k(x, x')] + E[k(y, y')] - 2E[k(x, y)] $

where, $k(.,.)$ is the kernel function. 
$x, x'$ are independent samples from $X$,
and $y, y'$ are independent samples from $Y$.

\subsubsection{Kernel}
\label{sssec:kernel}

A common choice of kernel for MMD is the Gaussian (RBF) kernel:

$ k(x, y) = \exp\left(-\frac{||x - y||^2}{r}\right) $

Where, $r$ is chosen by designer



\subsection{L2 Regularization}
\label{ssec:l2regularization}

L2 regularization introduces a penalty on the squared magnitudes of model weights. 
In Nerual Networks, the loss function with L2 regularization is:

$ J(\theta) = \text{MMD}(\theta) + \lambda ||\theta||_2^2 $

Where,

- $\lambda$ is the regularization strength

- $||\theta||_2^2$ is the L2 norm of the parameter vector.

\section{RESULTS}
\label{sec:results}

The paper title (on the first page) should begin 1.38 inches (35 mm) from the
top edge of the page, centered, completely capitalized, and in Times 14-point,
boldface type.  The authors' name(s) and affiliation(s) appear below the title
in capital and lower case letters.  Papers with multiple authors and
affiliations may require two or more lines for this information. Please note
that papers should not be submitted blind; include the authors' names on the
PDF.

\section{CONCLUSION}
\label{sec:conclusion}

To achieve {\bf not} the best rendering both in printed proceedings and electronic proceedings, we
strongly encourage you to use Times-Roman font.  In addition, this will give
the proceedings a more uniform look.  Use a font that is no smaller than nine
point type throughout the paper, including figure captions.

In nine point type font, capital letters are 2 mm high.  {\bf If you use the
smallest point size, there should be no more than 3.2 lines/cm (8 lines/inch)
vertically.}  This is a minimum spacing; 2.75 lines/cm (7 lines/inch) will make
the paper much more readable.  Larger type sizes require correspondingly larger
vertical spacing.  Please do not double-space your paper.  TrueType or
Postscript Type 1 fonts are preferred.

The first paragraph in each section should not be indented, but all the
following paragraphs within the section should be indented as these paragraphs
demonstrate.

\section{STATEMENT OF ALL TOOLS USED}
\label{sec:statementofalltoolsused}

Major headings, \cite{Lamp86} for example, "1. Introduction", should appear in all capital
letters, bold face if possible, centered in the column, with one blank line
before, and one blank line after. Use a period (".") after the heading number,
not a colon.

\subsection{Subheadings}
\label{ssec:subhead}

Subheadings should appear in lower case (initial word capitalized) in
boldface.  They should start at the left margin on a separate line.
 
\subsubsection{Sub-subheadings}
\label{sssec:subsubhead}

Sub-subheadings, as in this paragraph, are discouraged. However, if you
must use them, they should appear in lower case (initial word
capitalized) and start at the left margin on a separate line, with paragraph
text beginning on the following line.  They should be in italics.

% To start a new column (but not a new page) and help balance the last-page
% column length use \vfill\pagebreak.
% -------------------------------------------------------------------------
%\vfill
%\pagebreak

\vfill\pagebreak

% References should be produced using the bibtex program from suitable
% BiBTeX files (here: strings, refs, manuals). The IEEEbib.bst bibliography
% style file from IEEE produces unsorted bibliography list.
% -------------------------------------------------------------------------
\bibliographystyle{IEEEbib}
\bibliography{strings,refs}

\end{document}
